\chapter{类型安全}

绝大部分编程语言是安全的(类型安全,或强类型的。非正式地说,这意味着执行过程

中不会出现某些不匹配。举例说明:规定$\mathrm{E}$类型安全,绝对不会出现数字加上字符串,或连

接两个数字的情况。这两个操作都不是有意义的。

类型安全一般表述静态和动态的一致性。静态可以看成预计一个表达式的值具有某个形

式,以使动态下的这个表达式是良定义的。因此,分析不能“卡在”在一个无可能转换的状态

下,在实现上对应执行时“非法指令”错误的缺失。通过显示每一步保留可类型化的转换,或

显示可类型化的规定是良定义的,安全性可以得到证明。因此,分析绝不会“消失”,也绝不

会遇到非法指令。

对于语言$\mathrm{E}$的类型安全规定如下:

定理 6.1 (类型安全).

{\it 1. 如果} $e:\tau  且e\mapsto e'$, {\it 那么} $e'$ : $\tau.$

2. {\it 如果} $ e:\tau$, {\it 那么} $e\iota/\mathrm{a}/$, {\it 或存在} $e'$ {\it 满足} $e\mapsto e'.$

第一条称为{\it 保留性},指分析的每一步能保持类型的归类;第二条称作{\it 进展性},保证表达式

要么已经是值,要么可以进一步分析。安全性是以上两者的结合。

我们认为表达式$e$是{\it 中止的} 如果e不是值,且没有$e'$使$e\mapsto e'$. 成立。根据安全性定理,中

止的状态一定是弱类型的。换种说法,良类型的状态不会中止。

6.1 保留性

关于在第 4 章和第 5 章中定义的$\mathrm{E}$ 的保留性理论已通过在转换系统上使用归纳法(规则


5.4)得到证明。

定理  6.2 (保留性. {\it 如果} $e$ : $\tau$ {\it 且} $e\mapsto e'$, {\it 那么} $e'$ : $\tau.$

{\it 证明}. 给出两种情况的证明,其余留给读者。考虑规则5.4b $()$ ,

.
$$
\frac{e_{1}\mapsto e_{1}'}{\mathrm{p}\mathrm{l}\mathrm{u}\mathrm{s}(e_{1};e_{2})\mapsto \mathrm{p}1\mathrm{u}\mathrm{s}(e_{1}';e_{2})}
$$
假设 $(e_{1;}\cdot e_{2})$ : $\tau$. 通过类型转化,我们有 $\tau=\mathrm{n}\mathrm{u}\mathrm{m}, e_{1}$ : num, 和 $e_{2}$ : num.

归纳可得 eí : num, 且由此有 plus(eí; $e_{2}$) : num. 连接操作的处理与此相似。

现在考虑规则5.4h $()$ ,
$$
\overline{1\mathrm{e}\mathrm{t}(e_{1;}\cdot x.e_{2})\mapsto[e_{1}/x]e_{2}}\ 
$$
假设 let $(e_{1;}.\ x.e2)$ : $\tau_{2}$. 由引理4.2 , 设$e_{1}$ : $\tau_{1}$ , 故有 $x$: $\tau_{1} \vdash e_{2}$:

$\tau_{2}$. 通过替换引理4.4中的 $[e_{1}/x]e_{2}$ : $\tau_{2}$, 可以得证。

很容易得到原语操作全是类型保留的。比如, 如果 $a$ nat

且 $b$ nat 且 $a+b=c$, 那么 $c$ nat.
$$
\square 
$$
保留性的证明由转化形式判断上的归纳构成,因为它的论证在于对一个表达式所有可能转化

形式的检测。有时候我们会尝试通过对$e$的结构归纳,或类型归纳来得到证明,但经验告诉


我们,这往往会导致得不到想要的结论,或者根本不能付诸操作。


6.2进展性

进展性理论持有的观点是良类型程序不会 “中止”。

证明的关键是以下引理,它们描述了每种类型的值。

引理 6.3 (规范形式). {\it 如果} $e\iota/\mathrm{a}/且 e$ : $\tau$, {\it 那么}

{\it 1. 如果} $\tau=\mathrm{n}\mathrm{u}\mathrm{m}$, {\it 那么对某个数字} $n.$, $e=\mathrm{n}\mathrm{u}\mathrm{m}[n]$ 

2. {\it 如果} $\tau=\mathrm{s}\mathrm{t}\mathrm{r}$, {\it 那么对某个字符串} $\mathrm{s}.$,$e=$ str $[\mathrm{s}]$ 

{\it 证明}. 在规则(4.1)和(5.3)上归纳。 . $\square $

进展性通过归纳规则(4.1)定义语言的静态得证。

定理 6.4 (进展性. {\it 如果} $ e:\tau$, {\it 那么或者} $e\iota/\mathrm{a}/$, {\it 或者存在} $e'$ {\it 使} $e\mapsto e'.$

{\it 证明}. 证明通过类型推导归纳进行,只考虑一种情况,对于规则(4.1d),

$$
\frac{e_{1}:\mathrm{n}\mathrm{u}\mathrm{m}e_{2}:\mathrm{n}\mathrm{u}\mathrm{m}}{\mathrm{p}1\mathrm{u}\mathrm{s}(e_{1};e_{2}):\mathrm{n}\mathrm{u}\mathrm{m}}\ '
$$

上下文为空,因为我们不考虑闭项

归纳可得 $e_{1}$ 是值, 或存在 eí 使 $e_{1} \mapsto$eí. 在后一种情况下,根据要求,它遵循 plus $(e_{1;}\cdot e_{2}) \mapsto$plus (eí; $e_{2}$), 在前一种情况下,同样归纳得到 $e_{2}$ 是值, 或存在 $e_{2}'$ 使 $e_{2} \mapsto e_{2}'$. 后一情况,有 plus $(e_{1;}\cdot e_{2}) \mapsto$

plus $(e_{1;}\cdot e_{2}')$ 。 对于前一情况,通过
引理6.3,有 $e_{1}=\mathrm{n}\mathrm{u}\mathrm{m}[n_{1}]$

和 $e_{2}=$ num[{\it n}2], 因此

plus(num $[n_{1}]_{;}$. num[{\it n}2]) $\mapsto \mathrm{n}\mathrm{u}\mathrm{m}[n_{1}+n_{2}].$
$$
\square 
$$

因为表达式的类型规则是语法制导的,所以进展性定理可以同样由对 $e$ 的结构归纳来证

明,每一步都要使用反演定理。但这种方法在非语法制导的类型规则下不适用,也就是说,

一个给定的表达式形式有多个规则。这样的规则没有任何困难,只要证明是在类型规则上归

纳,而不是在表达式的结构上。

总之,保留性与进展性的结合构成安全的证明。进展定理确保了良类型的表达式不会在弱

定义的状态下“卡住”,而保留定理则确保了步骤执行后,结果仍是良类型的(类型保持相

同)。因此,这两个部分共同工作,以确保静态和动态是一致的,并且在评估良类型的表达

式时不会遇到弱定义的状态。

6.3 运行时错误

S假设我们想用一个除0 无意义的商运算扩展 $\mathrm{E}$, 商的自然类型规则由以下规则给出:
$$
\frac{e_{1}:\mathrm{n}\mathrm{u}\mathrm{m}e_{2}:\mathrm{n}\mathrm{u}\mathrm{m}}{\mathrm{d}\mathrm{i}\mathrm{v}(e_{1};e_{2}):\mathrm{n}\mathrm{u}\mathrm{m}}\ .
$$
但表达式 $\mathrm{d}\mathrm{i}\mathrm{v}$(num[3]; num $[0]$) 是良类型的,却中止了!有两个选项去纠正这个错误:

1. 增强类型系统,使任何良类型程序都不能除以0.

2. 添加动态检查,将以零为除数的除法作为评估结果的一个错误信号.

两个选择都是原则上可行的,但最常见的方法是第二种。第一条要求类型检查器证明表达式

为非零,然后才允许在一个商的分母中使用它。要做到这点必须将很多程序划为弱定义的。

我们不能静态地预测表达式在计算时是否为非零,因此第二种方法在实际中最常用。

总体思路是区分 {\it 检测错误} 和 {\it 非检测错误}。一个非检测错误是被类型系统排除的错误。不采

取运行时检测以确保不会发生这种错误,因为类型系统排除了发生这种错误的可能性。例如,

动态不需要检查,当执行加法时,它的两个参数实际上是数字,而不是字符串,因为类型系

统的保证。另一方面,商的动态必须检查零除数,因为类型系统没有排除这种可能性。

对检测错误进行建模的一种方法是给出判断

$e$ 的归纳定义,指出表达式 $e$ 可能检测出的运行时错误,如除以0。以下是一些具有代表性的规则,将在此判断的完全归纳定义中出现:
\begin{center}
$\displaystyle \frac{e_{1}\mathrm{v}\mathrm{a}1}{\mathrm{d}\mathrm{i}\mathrm{v}(e_{1};\mathrm{n}\mathrm{u}\mathrm{m}[0])\mathrm{e}\mathrm{r}\mathrm{r}}$   (6.1a)

$\displaystyle \frac{e_{1}\mathrm{e}\mathrm{r}\mathrm{r}}{\mathrm{d}\mathrm{i}\mathrm{v}(e_{1};e_{2})\mathrm{e}\mathrm{r}\mathrm{r}}$   (6.1b)

$\displaystyle \frac{e_{1}\mathrm{v}\mathrm{a}1e_{2}\mathrm{e}\mathrm{r}\mathrm{r}}{\mathrm{d}\mathrm{i}\mathrm{v}(e_{1};e_{2})\mathrm{e}\mathrm{r}\mathrm{r}}$   (6.1c)
\end{center}
规则(6.1a)表示除零错误。其他规则向上传播这个错误:  如果一个子表达式分析是检测错
误,那么整个表达式同样。

一旦有了错误判断,我们还可以考虑一个表达式错误。它强制引入一个错误,使用以下静
态和动态语义: 

(6.2a)

$\Gamma\vdash$ error : $\tau$

(6.2b)
$$
\overline{\mathrm{e}\mathrm{r}\mathrm{r}\mathrm{o}\mathrm{r}\mathrm{e}\mathrm{r}\mathrm{r}}
$$
保留定理不受检测错误的影响。但进展性的声明与证明需要修改。



定理 6.5 (考虑错误的进展性). {\it 如果} $e$ : $\tau$, {\it 那么或者} $e$ {\it err, 或者} $e \iota/\mathrm{a}/$, {\it 或存在} $e'$ {\it 使}$e\mapsto e'.$

{\it 证明}. 通过类型归纳证明,与前面给出的证明类似,只是在每一点的证明上需要考虑三个情况。 $\square $

6.4 小结

类型安全的概念最初是由 $()$ 提出的, 他提出了 ``良定义的程序不会出错'' 。Milner relied on 依靠一个明确的“出错”的概念来表达类型错误的概念,而 $()$ 观察到
我们可以相反地表明,在一个良类型程序中不可能弱定义的状态,这就产生了“良类型化程序不会被卡住”的说法。

然而,他们的提论依赖于一项表明没有一个中止状态是良类型的分析。进展理论,依赖于 $()$ , 风格规范形式的描述,消除了这一分析。

习题: 
6.1.完成定理6.2 的详细证明 
6.2 完成定理6.4 的详细证明 
6.3.  给出定理6.5 的几个例子,来说明在类型安全证明中是如何处理检测错误的。


